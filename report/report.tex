\documentclass[12pt]{IEEEtran}

\usepackage[utf8]{inputenc}
\usepackage{amsmath}
\usepackage{units}
\usepackage{graphicx}
\usepackage{multirow}
\usepackage{comment}
\usepackage{listings}
\usepackage{placeins}

\begin{document}
\title{A Multi-modal approach to Human Action Classifications}

\author{\IEEEauthorblockN{Tan Ren Jie\IEEEauthorrefmark{1},
Kelvin Yeo Ngan Chong \IEEEauthorrefmark{2}, and Khaing Mon Kyaw\IEEEauthorrefmark{3} } \\
\IEEEauthorblockA{Institute of Systems Science, National University of Singapore \\
Email: \IEEEauthorrefmark{1}e0267395@u.nus.edu,
\IEEEauthorrefmark{2}e0267573@u.nus.edu,
\IEEEauthorrefmark{3}e0267573@u.nus.edu,
}}
\maketitle

\begin{abstract}

\end{abstract}

\section{Introduction}
\begin{itemize}
\item Open with background infomration of the problem case
\item Present literature review of other works and their limitations
\item Summary of how your work will tackle the limitations of others/solve the problem
\end{itemize}

\section{Methods and Modelling}
\begin{itemize}
\item Present the derivation of your mdoel and explanations
\end{itemize}
\section{Computational Simulation}
\begin{itemize}
\item Present the data source and programme/simulation to be conducted
\end{itemize}
\section{Results}
\begin{itemize}
\item Present the results of your model simulation
\end{itemize}
\section{Dicussion}
\begin{itemize}
\item Compare your results with other models and highlight its effectiveness
\item Dicuss implications, limitations and future work
\end{itemize}
\section{Conclusion}

\end{document}
